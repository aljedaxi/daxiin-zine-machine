\documentclass[12pt, openany]{book}
    \usepackage[final]{pdfpages}
    \usepackage{newclude}
	\usepackage{hyperref}
	\usepackage{longtable} %TODO maybe?
	\usepackage{booktabs} %TODO maybe?
	%\usepackage[paper=letter,DIV=10,BCOR=5mm]{typearea}
	\BLOCK{if printing == 'booklet'}
	\usepackage[
		 headinclude=false
		,footinclude=false
		,pagesize
		,twoside
		,paper=letter
		,DIV=15
		,BCOR=5mm
	]{typearea}
	\BLOCK{endif}
	\usepackage{tocloft}
	\usepackage{alltt}
	\usepackage{courier}
%fonts {
	\usepackage[normalem]{ulem}
	\usepackage{microtype}
	%\usepackage[cmintegrals,cmbraces]{newtxmath}
	%\usepackage{ebgaramond-maths}
	\usepackage{\VAR{g_font}}
	\usepackage[T1]{fontenc}
	%\usepackage{yfonts}
%}


%images {
	\usepackage{graphicx}
	\usepackage{float}
%}

%poetica {
	\usepackage{fancyhdr,ifthen,keyval,poemscol}
	\renewenvironment{poem}[1][\linewidth]
	{\raggedright%
	\large
	\language=255%no hyphenation in verse
	\noemendyettrue%
	\noexplainyettrue%
	\setcounter{verselinenumber}{0}\setcounter{printlineindex}{0}%
	\nobreak\begin{pmclverse}[#1]%
	\inpoemtrue\nobreak\mark{\relax}%
	}{\end{pmclverse}%
	\inpoemfalse\mark{\relax}%
	\goodbreak\afterpoemskip%\bigskip
	}
	\setverselinemodulo{2}
	
	\pagestyle{fancy}
	\cfoot{}
%}

%permeate formatting {
	\newlength{\tocwidth}
	\setlength{\tocwidth}{12em}

	\newcommand{\daxiincontent}[2]{
		\addcontentsline{toc}{section}{
			#1
		}
	}
	\newcommand{\articletitle}[1]{
		{\bf #1}
	}
	\newcommand{\articlesubtitle}[1]{
		{\bf #1}
	}
	\newcommand{\articleauthor}[1]{
		---#1
	}
	\newcommand{\articlebio}[1]{
		#1
	}
	\newcommand{\articlerights}[1]{

		This article is released under #1.
	}
%}

%header {
	\leftheader{\VAR{g_title} --- \VAR{g_author}}

	\newcommand{\daxiinhead}[1]{%2]{
		\rightheader{
			#1 %--- #2
		}
	}
%}

%Jinja2 variables {
	\title{\VAR{g_title}}
	\author{\VAR{g_author}}
%}

\begin{document}
\frontmatter
\Large

\BLOCK{if frontcover}
	\includepdf{\VAR{frontcover}}
\BLOCK{else}
	\maketitle
\BLOCK{endif}

\BLOCK{if TOC}
	\tableofcontents
\BLOCK{endif}

\mainmatter

\BLOCK{for filename in files}
	\include{\VAR{filename}}
\BLOCK{endfor}

\BLOCK{if backcover}
	\includepdf{\VAR{backcover}}
\BLOCK{endif}

\end{document}
%/*
	%\frontmatter
	%\includepdf{cover}
	%\topskip0pt
	%    \vspace*{\fill}
	%    \noindent
	%    \begin{poem}
	%    Dies ist kein Buch: was liegt an B\"uchern! \verseline
	%    An diesen S\"argen und Leichent\"uchern!\verseline
	%    Vergangnes ist der B\"ucher Beute:\verseline
	%    Doch hierin lebt ein ewig \emph{Heute}.
	%    \end{poem}
	%
	%    ~\\
	%
	%    {\large \noindent Creative Commons NonCommercial--ShareAlike 4.0 International \\(CC BY-NC-SA 4.0); 2018; \\jacob whitford---bender\&Friedrich Nietzsche. 
	%
	%    ~\\ \addcontentsline{toc}{section}{\makebox[16em][l]{Die fr\"oliche Wissenschaft} (1882)}\makebox[6in][r]{/x,$\beta$}
	%
	%    \noindent   dedicated to my own \emph{Salom\'e}---of sorts---\\
	%                who treats me now like the tide on a moonless night.}
	%
	%    ~\\
	%
	%    \begin{poem}
	%    This is no book: what lies within them!\verseline
	%    Within these shrouds of the cold \& condemn'd!\verseline
	%    Departed days are a Book's best prey:\verseline
	%    But herein lives an eternal \emph{today}.
	%    \end{poem}
	%    \vspace*{\fill}
	%\tableofcontents
	%\newpage
	%    \section*{Translator's note}
	%    Firstly, it is worth noting that `Keikaku' means plan.
	%
	%    Special thanks to the Nietzsche Channel, who put all the German poems in one place. That's a good place to look if you want a more literal translation. Kaufmann is also always good.
	%
	%    Generally, these are in chronological order, but i really like Prinz Vogelfrei. The best for last.
	%
	%    this ain't your everyday translation. This is meant to be \emph{poetry}: it's meant to look and feel and taste like poetry: the kind of poetry Nietzsche wrote. %for the vast majority of these,
	%    Every syllable that's stressed in German is stressed in English; every line has the same number of syllables, and every line that rhymes in German, rhymes in English.
	%
	%    Well, stress is a difficult thing to quantify. Take that first claim gently.
	%
	%    To pull this off, i had to play a bit more fast and loose with the language. There are some lines that have a---well, somewhat radically---different meaning in English than they do in German. Obviously, Nietzsche never wrote about Napalm. He wrote about gun smoke. However, the poem's overall meaning should be upheld.
	%
	%    In this sense, it should be understood that these poems are by Nietzsche, \emph{and} [alje]daxi. My job was something more than translation. I am translator, editor and, to a very real extent, writer. When i say radically different i mean Nietzsche never talked about `cloud-skin'. His poetry is often quite straightforward.
	%
	%    If you want to read something quite straightforwards, i'd recommend Kaufmann. He's very good at translating prose. But i don't want poetry in, prose out. I want to write \emph{poetry, dammit}---and i'll take liberties with the literal meanings of words if it means the actual \emph{poetry} of the poem comes out.
	%
	%    Anyways, i doubt Nietzsche would mind that i'm pumping a little life back into his old verse.
	%
	%    ---daxi.
	%\mainmatter
	%\vspace*{\fill}
	%\input{kein_buch}
	%\vspace*{\fill}
	%\addcontentsline{toc}{section}{\makebox[16em][l]{Ohne Heimath} (1859)}
	%    \rightheader{Ohne Heimath (1859)}
	%    \vspace*{\fill}
	%    \poemtitle{\textfrak{Ohne Heimath}}
	%    \begin{poem}
	%    \begin{stanza}
	%    Flücht'ge Rosse tragen\verseline
	%    mich ohn' Furcht und Zagen\verseline
	%    durch die weite Fern.\verseline
	%    Und wer mich sieht, der kennt mich,\verseline
	%    und wer mich kennt, der nennt mich\verseline
	%    den heimatlosen Herrn.\verseline
	%    Heidideldi!\verseline
	%    Verlaß mich nie,\verseline
	%    mein Glück, du heller Stern!
	%    \end{stanza}
	%    \begin{stanza}
	%    Niemand darf es wagen,\verseline
	%    mich darnach zu fragen,\verseline
	%    wo meine Heimat sei.\verseline
	%    Ich bin wohl nie gebunden\verseline
	%    an Raum und flücht'ge Stunden,\verseline
	%    bin wie der Aar so frei.\verseline
	%    Heidideldi!\verseline
	%    Verlaß mich nie,\verseline
	%    mein Glück, du holder Mai!
	%    \end{stanza}
	%    \begin{stanza}
	%    Daß ich einst soll sterben,\verseline
	%    küssen muß den herben\verseline
	%    Tod, das glaub ich kaum.\verseline
	%    Zum Grabe soll ich sinken\verseline
	%    und nimmermehr dann trinken\verseline
	%    des Lebens duft'gen Schaum?\verseline
	%    Heidideldi!\verseline
	%    Verlaß mich nie,\verseline
	%    mein Glück, du bunter Traum!
	%    \end{stanza}
	%    \end{poem}
	%    \poemtitle{Without a Home}
	%    \begin{poem}
	%    \begin{stanza}
	%    Fleeting horses carry\verseline
	%    Me, polite and carefree\verseline
	%    Through the wide afar.\verseline
	%    Who sees me now, should know me,\verseline
	%    Who knows me now, should call me:\verseline
	%    The Homeland-lacking heart.\verseline
	%    Heidideldi!\verseline
	%    Remain with me,\verseline
	%    my joy, you dazzling star!
	%    \end{stanza}
	%    \begin{stanza}
	%    None should ever dare it,\verseline
	%    To take their qualm, and air it,\verseline
	%    Just where my home might be.\verseline
	%    I've not once been committed\verseline
	%    to places or fickle minutes,\verseline
	%    I'm---as the eagle---free.\verseline
	%    Heidideldi!\verseline
	%    Remain with me,\verseline
	%    my joy, you May in glee!
	%    \end{stanza}
	%    \begin{stanza}
	%    That i should sometime die\verseline
	%    Makes kisses taste like lye\verseline
	%    Death, i doubt its scheme.\verseline
	%    Unto the Grave i'll sink\verseline
	%    and nevermore i'll drink\verseline
	%    life's most delicate cream?\verseline
	%    Heidideldi!\verseline
	%    Remain with me,\verseline
	%    my joy, you vibrant dream!
	%    \end{stanza}
	%    \end{poem}
	%    \vspace*{\fill}
	%%\include[\vspace*{\fill}]{im_basel}[\vspace*{\fill}]
	%\vspace*{\fill}
	%%\input{melancholia.tex}
	%\vspace*{\fill}
	%%\input{gewitter.tex}
	%\vspace*{\fill}\newpage
	%\vspace*{\fill}
	%\addcontentsline{toc}{section}{\makebox[16em][l]{Zarathustra's Rundgesang} (1882)}
	%\rightheader{Zarathustra's Rundgesang (1882)}
	%	\poemtitle{\textfrak{Zarathustras Rundgesang}}
	%	\begin{poem}
	%	\begin{stanza}
	%	O Mensch! Gib acht!\verseline
	%	Was spricht die tiefe Mitternacht?\verseline
	%	>>Ich schlief, ich schlief---,\verseline
	%	Aus tiefem Traum bin ich erwacht:---\verseline
	%	Die Welt ist tief,\verseline
	%	Und tiefer als der Tag gedacht.\verseline
	%	Tief ist ihr Weh---,\verseline
	%	Lust—tiefer noch als Herzeleid:\verseline
	%	Weh spricht: Vergeh!\verseline
	%	Doch alle Lust will Ewigkeit---,\verseline
	%	---will tiefe, tiefe Ewigkeit!<<
	%	\end{stanza}
	%	\end{poem}
	%	\poemtitle{Zarathustra's Chorus Song}
	%	\begin{poem}
	%	\begin{stanza}
	%	O Man! Eyes bright!\verseline
	%	What says the deepest, dead of night?\verseline
	%	>>I sleep, I sleep---,\verseline
	%	From deepest sleep, i'm set aflame:---\verseline
	%	The world is deep,\verseline
	%	And deeper than the day could claim.\verseline
	%	Deep is your pain---,\verseline
	%	Joy's deeper still than pain can be:\verseline
	%	Pain saith: Decay!\verseline
	%	But all Joy wills eternity---,\verseline
	%	---the depths, of deep eternity!<<
	%	\end{stanza}
	%	\end{poem}
	%\vspace*{\fill}\newpage
	%\vspace*{\fill}
	%\addcontentsline{toc}{section}{\makebox[16em][l]{Pinie und Blitz} (1882)}
	%\leftheader{Pinie und Blitz (1882)}
	%	\poemtitle{\textfrak{Pinie und Blitz}}
	%	\begin{poem}
	%	\begin{stanza}
	%	Hoch wuchs ich \"uber Mensch und Tier;\verseline
	%	Und sprech ich---niemand spricht mit mir.
	%	\end{stanza}
	%	\begin{stanza}
	%	Zu einsam wuchs ich und zu hoch---\verseline
	%	Ich warte: worauf wart' ich doch?
	%	\end{stanza}
	%	\begin{stanza}
	%	Zu nah ist mir der Wolken Sitz,---\verseline
	%	Ich warte auf den ersten Blitz.
	%	\end{stanza}
	%	\end{poem}
	%	\poemtitle{Lightning and Pine}
	%	\begin{poem}
	%	\begin{stanza}
	%	'Bove man and plant, I grew so high\verseline
	%	Now when I speak---none can reply.
	%	\end{stanza}
	%	\begin{stanza}
	%	I grew too lonely with my height---\verseline
	%	I wait; for what to fill my sight?
	%	\end{stanza}
	%	\begin{stanza}
	%	So close are clouds that herald storm,---\verseline
	%	A wildfire's to be born.
	%	\end{stanza}
	%	\end{poem}
	%\vspace*{\fill}
	%\newpage
	%\vspace*{\fill}
	%\addcontentsline{toc}{section}{\makebox[16em][l]{Eis} (1882)}
	%\leftheader{Eis (1882)}
	%	\poemtitle{\textfrak{Eis}}
	%	\begin{poem}
	%	Ja! Mitunter mach' ich Eis:\verseline
	%	Nützlich ist Eis zum Verdauen!\verseline
	%	Hättet ihr viel zu verdauen,\verseline
	%	Oh wie liebtet ihr mein Eis!\verseline
	%	\end{poem}
	%
	%	\poemtitle{Ice}
	%	\begin{poem}
	%	Yuh! Now, sometimes i make ice:\verseline
	%	How helpful ice is for digestion!\verseline
	%	And y'all have so much for digestion,\verseline
	%	O, how love you so, my ice!\verseline
	%	\end{poem}
	%\vspace*{\fill}
	%\newpage
	%\vspace*{\fill}
	%\addcontentsline{toc}{section}{\makebox[16em][l]{Heraklitismus} (1882)}
	%\rightheader{Heraklitismus (1882)}
	%	\poemtitle{\textfrak{Heraklitismus}}
	%	\begin{poem}
	%    Alles Glück auf Erden,\verseline
	%    Freunde, giebt der Kampf!\verseline
	%    Ja, um Freund zu werden,\verseline
	%    Braucht es Pulverdampf!\verseline
	%    Eins in Drei'n sind Freunde:\verseline
	%    Brüder vor der Noth,\verseline
	%    Gleiche vor dem Feinde,\verseline
	%    Freie—vor dem Tod!\verseline
	%	\end{poem}
	%
	%	\poemtitle{Heraclitus-ism}
	%	\begin{poem}
	%    All that's grand and earthly,\verseline
	%    Friends, give for the fight!\verseline
	%        %translating Kampf
	%    Aye, should we be friendly,\verseline
	%    It'll be by napalm light!\verseline
	%        %original: gunsmoke
	%    One in three are friendly:\verseline
	%    Brothers jointly fend,\verseline
	%    Likewise for the fiendly,\verseline
	%    Freely---'fore their end!
	%	\end{poem}
	%\vspace*{\fill}
	%%\input{wort.tex}
	%\vspace*{\fill}
	%%\input{kleine_hexe}
	%\vspace*{\fill}
	%\input{prinz_vogelfrei}\newpage
	%\vspace*{\fill}
	%\addcontentsline{toc}{section}{\makebox[16em][l]{Wer viel einst zu verk\"unden hat} (1883)}
	%\rightheader{Wer viel einst zu verk\"unden hat (1883)}
	%	\poemtitle{\textfrak{Wer viel einst zu verk\"unden hat}}
	%	\begin{poem}
	%	\begin{stanza}
	%	Wer viel einst zu verk\"unden hat,\verseline
	%	Schweigt viel in sich hinein:\verseline
	%	Wer einst den Blitz zu z\"unden hat,\verseline
	%	Mu\ss~lange---Wolke sein.
	%	\end{stanza}
	%	\end{poem}
	%	\poemtitle{When one has plenty great to tell.}
	%	\begin{poem}
	%	\begin{stanza}
	%	When one has plenty great to tell,\verseline
	%	One hones silence within:\verseline
	%	When one has lightning to dispel,\verseline
	%	One long hones a cloud's skin.
	%	\end{stanza}
	%	\end{poem}
	%\vspace*{\fill}
	%\include[\vspace*{\fill}]{geht_die}[\vspace*{\fill}]
	%\vspace*{\fill}
	%%\input{freigeist.tex}
	%\vspace*{\fill}
	%\input{an_spinoza}
	%\vspace*{\fill}
	%\newpage
	%\vspace*{\fill}
	%\input{ehere}
	%\vspace*{\fill}
	%\input{letzter_wille}
	%\vspace*{\fill}
	%\end{document}
%*/
