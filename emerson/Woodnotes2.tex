WOODNOTES II

_As sunbeams stream through liberal space_
_And nothing jostle or displace,_
_So waved the pine-tree through my thought_
_And fanned the dreams it never brought._

'Whether is better, the gift or the donor?
Come to me,'
Quoth the pine-tree,
'I am the giver of honor.
My garden is the cloven rock,
And my manure the snow;
And drifting sand-heaps feed my stock,
In summer's scorching glow.
He is great who can live by me:
The rough and bearded forester
Is better than the lord;
God fills the script and canister,
Sin piles the loaded board.
The lord is the peasant that was,
The peasant the lord that shall be;
The lord is hay, the peasant grass,
One dry, and one the living tree.
Who liveth by the ragged pine
Foundeth a heroic line;
Who liveth in the palace hall
Waneth fast and spendeth all.
He goes to my savage haunts,
With his chariot and his care;
My twilight realm he disenchants,
And finds his prison there.

'What prizes the town and the tower?
Only what the pine-tree yields;
Sinew that subdued the fields;
The wild-eyed boy, who in the woods
Chants his hymn to hills and floods,
Whom the city's poisoning spleen
Made not pale, or fat, or lean;
Whom the rain and the wind purgeth,
Whom the dawn and the day-star urgeth,
In whose cheek the rose-leaf blusheth,
In whose feet the lion rusheth,
Iron arms, and iron mould,
That know not fear, fatigue, or cold.
I give my rafters to his boat,
My billets to his boiler's throat,
And I will swim the ancient sea
To float my child to victory,
And grant to dwellers with the pine
Dominion o'er the palm and vine.
Who leaves the pine-tree, leaves his friend,
Unnerves his strength, invites his end.
Cut a bough from my parent stem,
And dip it in thy porcelain vase;
A little while each russet gem
Will swell and rise with wonted grace;
But when it seeks enlarged supplies,
The orphan of the forest dies.
Whoso walks in solitude
And inhabiteth the wood,
Choosing light, wave, rock and bird,
Before the money-loving herd,
Into that forester shall pass,
From these companions, power and grace.
Clean shall he be, without, within,
From the old adhering sin,
All ill dissolving in the light
Of his triumphant piercing sight:
Not vain, sour, nor frivolous;
Not mad, athirst, nor garrulous;
Grave, chaste, contented, though retired,
And of all other men desired.
On him the light of star and moon
Shall fall with purer radiance down;
All constellations of the sky
Shed their virtue through his eye.
Him Nature giveth for defence
His formidable innocence;
The mounting sap, the shells, the sea,
All spheres, all stones, his helpers be;
He shall meet the speeding year,
Without wailing, without fear;
He shall be happy in his love,
Like to like shall joyful prove;
He shall be happy whilst he wooes,
Muse-born, a daughter of the Muse.
But if with gold she bind her hair,
And deck her breast with diamond,
Take off thine eyes, thy heart forbear,
Though thou lie alone on the ground.

'Heed the old oracles,
Ponder my spells;
Song wakes in my pinnacles
When the wind swells.
Soundeth the prophetic wind,
The shadows shake on the rock behind,
And the countless leaves of the pine are strings
Tuned to the lay the wood-god sings.
  Hearken! Hearken!
If thou wouldst know the mystic song
Chanted when the sphere was young.
Aloft, abroad, the paean swells;
O wise man! hear'st thou half it tells?
O wise man! hear'st thou the least part?
'Tis the chronicle of art.
To the open ear it sings
Sweet the genesis of things,
Of tendency through endless ages,
Of star-dust, and star-pilgrimages,
Of rounded worlds, of space and time,
Of the old flood's subsiding slime,
Of chemic matter, force and form,
Of poles and powers, cold, wet, and warm:
The rushing metamorphosis
Dissolving all that fixture is,
Melts things that be to things that seem,
And solid nature to a dream.
O, listen to the undersong,
The ever old, the ever young;
And, far within those cadent pauses,
The chorus of the ancient Causes!
Delights the dreadful Destiny
To fling his voice into the tree,
And shock thy weak ear with a note
Breathed from the everlasting throat.
In music he repeats the pang
Whence the fair flock of Nature sprang.
O mortal! thy ears are stones;
These echoes are laden with tones
Which only the pure can hear;
Thou canst not catch what they recite
Of Fate and Will, of Want and Right,
Of man to come, of human life,
Of Death and Fortune, Growth and Strife.'

  Once again the pine-tree sung:--
'Speak not thy speech my boughs among:
Put off thy years, wash in the breeze;
My hours are peaceful centuries.
Talk no more with feeble tongue;
No more the fool of space and time,
Come weave with mine a nobler rhyme.
Only thy Americans
Can read thy line, can meet thy glance,
But the runes that I rehearse
Understands the universe;
The least breath my boughs which tossed
Brings again the Pentecost;
To every soul resounding clear
In a voice of solemn cheer,--
"Am I not thine? Are not these thine?"
And they reply, "Forever mine!"
My branches speak Italian,
English, German, Basque, Castilian,
Mountain speech to Highlanders,
Ocean tongues to islanders,
To Fin and Lap and swart Malay,
To each his bosom-secret say.

  'Come learn with me the fatal song
Which knits the world in music strong,
Come lift thine eyes to lofty rhymes,
Of things with things, of times with times,
Primal chimes of sun and shade,
Of sound and echo, man and maid,
The land reflected in the flood,
Body with shadow still pursued.
For Nature beats in perfect tune,
And rounds with rhyme her every rune,
Whether she work in land or sea,
Or hide underground her alchemy.
Thou canst not wave thy staff in air,
Or dip thy paddle in the lake,
But it carves the bow of beauty there,
And the ripples in rhymes the oar forsake.
The wood is wiser far than thou;
The wood and wave each other know
Not unrelated, unaffied,
But to each thought and thing allied,
Is perfect Nature's every part,
Rooted in the mighty Heart,
But thou, poor child! unbound, unrhymed,
Whence camest thou, misplaced, mistimed,
Whence, O thou orphan and defrauded?
Is thy land peeled, thy realm marauded?
Who thee divorced, deceived and left?
Thee of thy faith who hath bereft,
And torn the ensigns from thy brow,
And sunk the immortal eye so low?
Thy cheek too white, thy form too slender,
Thy gait too slow, thy habits tender
For royal man;--they thee confess
An exile from the wilderness,--
The hills where health with health agrees,
And the wise soul expels disease.
Hark! in thy ear I will tell the sign
By which thy hurt thou may'st divine.
When thou shalt climb the mountain cliff,
Or see the wide shore from thy skiff,
To thee the horizon shall express
But emptiness on emptiness;
There lives no man of Nature's worth
In the circle of the earth;
And to thine eye the vast skies fall,
Dire and satirical,
On clucking hens and prating fools,
On thieves, on drudges and on dolls.
And thou shalt say to the Most High,
"Godhead! all this astronomy,
And fate and practice and invention,
Strong art and beautiful pretension,
This radiant pomp of sun and star,
Throes that were, and worlds that are,
Behold! were in vain and in vain;--
It cannot be,--I will look again.
Surely now will the curtain rise,
And earth's fit tenant me surprise;--
But the curtain doth _not_ rise,
And Nature has miscarried wholly
Into failure, into folly."

'Alas! thine is the bankruptcy,
Blessed Nature so to see.
Come, lay thee in my soothing shade,
And heal the hurts which sin has made.
I see thee in the crowd alone;
I will be thy companion.
Quit thy friends as the dead in doom,
And build to them a final tomb;
Let the starred shade that nightly falls
Still celebrate their funerals,
And the bell of beetle and of bee
Knell their melodious memory.
Behind thee leave thy merchandise,
Thy churches and thy charities;
And leave thy peacock wit behind;
Enough for thee the primal mind
That flows in streams, that breathes in wind:
Leave all thy pedant lore apart;
God hid the whole world in thy heart.
Love shuns the sage, the child it crowns,
Gives all to them who all renounce.
The rain comes when the wind calls;
The river knows the way to the sea;
Without a pilot it runs and falls,
Blessing all lands with its charity;
The sea tosses and foams to find
Its way up to the cloud and wind;
The shadow sits close to the flying ball;
The date fails not on the palm-tree tall;
And thou,--go burn thy wormy pages,--
Shalt outsee seers, and outwit sages.
Oft didst thou thread the woods in vain
To find what bird had piped the strain:--
Seek not, and the little eremite
Flies gayly forth and sings in sight.

'Hearken once more!
I will tell thee the mundane lore.
Older am I than thy numbers wot,
Change I may, but I pass not.
Hitherto all things fast abide,
And anchored in the tempest ride.
Trenchant time behoves to hurry
All to yean and all to bury:
All the forms are fugitive,
But the substances survive.
Ever fresh the broad creation,
A divine improvisation,
From the heart of God proceeds,
A single will, a million deeds.
Once slept the world an egg of stone,
And pulse, and sound, and light was none;
And God said, "Throb!" and there was motion
And the vast mass became vast ocean.
Onward and on, the eternal Pan,
Who layeth the world's incessant plan,
Halteth never in one shape,
But forever doth escape,
Like wave or flame, into new forms
Of gem, and air, of plants, and worms.
I, that to-day am a pine,
Yesterday was a bundle of grass.
He is free and libertine,
Pouring of his power the wine
To every age, to every race;
Unto every race and age
He emptieth the beverage;
Unto each, and unto all,
Maker and original.
The world is the ring of his spells,
And the play of his miracles.
As he giveth to all to drink,
Thus or thus they are and think.
With one drop sheds form and feature;
With the next a special nature;
The third adds heat's indulgent spark;
The fourth gives light which eats the dark;
Into the fifth himself he flings,
And conscious Law is King of kings.
As the bee through the garden ranges,
From world to world the godhead changes;
As the sheep go feeding in the waste,
From form to form He maketh haste;
This vault which glows immense with light
Is the inn where he lodges for a night.
What recks such Traveller if the bowers
Which bloom and fade like meadow flowers
A bunch of fragrant lilies be,
Or the stars of eternity?
Alike to him the better, the worse,--
The glowing angel, the outcast corse.
Thou metest him by centuries,
And lo! he passes like the breeze;
Thou seek'st in globe and galaxy,
He hides in pure transparency;
Thou askest in fountains and in fires,
He is the essence that inquires.
He is the axis of the star;
He is the sparkle of the spar;
He is the heart of every creature;
He is the meaning of each feature;
And his mind is the sky.
Than all it holds more deep, more high.'
