1

When the pine tosses its cones
To the song of its waterfall tones,
Who speeds to the woodland walks?
To birds and trees who talks?
Caesar of his leafy Rome,
There the poet is at home.
He goes to the river-side,--
Not hook nor line hath he;
He stands in the meadows wide,--
Nor gun nor scythe to see.
Sure some god his eye enchants:
What he knows nobody wants.
In the wood he travels glad,
Without better fortune had,
Melancholy without bad.
Knowledge this man prizes best
Seems fantastic to the rest:
Pondering shadows, colors, clouds,
Grass-buds and caterpillar-shrouds,
Boughs on which the wild bees settle,
Tints that spot the violet's petal,
Why Nature loves the number five,
And why the star-form she repeats:
Lover of all things alive,
Wonderer at all he meets,
Wonderer chiefly at himself,
Who can tell him what he is?
Or how meet in human elf
Coming and past eternities?

2

And such I knew, a forest seer,
A minstrel of the natural year,
Foreteller of the vernal ides,
Wise harbinger of spheres and tides,
A lover true, who knew by heart
Each joy the mountain dales impart;
It seemed that Nature could not raise
A plant in any secret place,
In quaking bog, on snowy hill,
Beneath the grass that shades the rill,
Under the snow, between the rocks,
In damp fields known to bird and fox.
But he would come in the very hour
It opened in its virgin bower,
As if a sunbeam showed the place,
And tell its long-descended race.
It seemed as if the breezes brought him,
It seemed as if the sparrows taught him;
As if by secret sight he knew
Where, in far fields, the orchis grew.
Many haps fall in the field
Seldom seen by wishful eyes,
But all her shows did Nature yield,
To please and win this pilgrim wise.
He saw the partridge drum in the woods;
He heard the woodcock's evening hymn;
He found the tawny thrushes' broods;
And the shy hawk did wait for him;
What others did at distance hear,
And guessed within the thicket's gloom,
Was shown to this philosopher,
And at his bidding seemed to come.

3

In unploughed Maine he sought the lumberers' gang
Where from a hundred lakes young rivers sprang;
He trode the unplanted forest floor, whereon
The all-seeing sun for ages hath not shone;
Where feeds the moose, and walks the surly bear,
And up the tall mast runs the woodpecker.
He saw beneath dim aisles, in odorous beds,
The slight Linnaea hang its twin-born heads,
And blessed the monument of the man of flowers,
Which breathes his sweet fame through the northern bowers.
He heard, when in the grove, at intervals,
With sudden roar the aged pine-tree falls,--
One crash, the death-hymn of the perfect tree,
Declares the close of its green century.
Low lies the plant to whose creation went
Sweet influence from every element;
Whose living towers the years conspired to build,
Whose giddy top the morning loved to gild.
Through these green tents, by eldest Nature dressed,
He roamed, content alike with man and beast.
Where darkness found him he lay glad at night;
There the red morning touched him with its light.
Three moons his great heart him a hermit made,
So long he roved at will the boundless shade.
The timid it concerns to ask their way,
And fear what foe in caves and swamps can stray,
To make no step until the event is known,
And ills to come as evils past bemoan.
Not so the wise; no coward watch he keeps
To spy what danger on his pathway creeps;
Go where he will, the wise man is at home,
His hearth the earth,--his hall the azure dome;
Where his clear spirit leads him, there's his road
By God's own light illumined and foreshowed.

4

'T was one of the charmèd days
When the genius of God doth flow;
The wind may alter twenty ways,
A tempest cannot blow;
It may blow north, it still is warm;
Or south, it still is clear;
Or east, it smells like a clover-farm;
Or west, no thunder fear.
The musing peasant, lowly great,
Beside the forest water sate;
The rope-like pine-roots crosswise grown
Composed the network of his throne;
The wide lake, edged with sand and grass,
Was burnished to a floor of glass,
Painted with shadows green and proud
Of the tree and of the cloud.
He was the heart of all the scene;
On him the sun looked more serene;
To hill and cloud his face was known,--
It seemed the likeness of their own;
They knew by secret sympathy
The public child of earth and sky.
'You ask,' he said, 'what guide
Me through trackless thickets led,
Through thick-stemmed woodlands rough and wide.
I found the water's bed.
The watercourses were my guide;
I travelled grateful by their side,
Or through their channel dry;
They led me through the thicket damp,
Through brake and fern, the beavers' camp,
Through beds of granite cut my road,
And their resistless friendship showed.
The falling waters led me,
The foodful waters fed me,
And brought me to the lowest land,
Unerring to the ocean sand.
The moss upon the forest bark
Was pole-star when the night was dark;
The purple berries in the wood
Supplied me necessary food;
For Nature ever faithful is
To such as trust her faithfulness.
When the forest shall mislead me,
When the night and morning lie,
When sea and land refuse to feed me,
'T will be time enough to die;
Then will yet my mother yield
A pillow in her greenest field,
Nor the June flowers scorn to cover
The clay of their departed lover.'
