\daxiincontent{The Snow-Storm}{Ralph Waldo Emerson}

\articletitle{The Snow-Storm}

\articleauthor{Ralph Waldo Emerson}
\begin{poem}
	\begin{stanza}
			Announced by all the trumpets of the sky,\verseline
			Arrives the snow, and, driving o'er the fields,\verseline
			Seems nowhere to alight: the whited air\verseline
			Hides hills and woods, the river, and the heaven,\verseline
			And veils the farm-house at the garden's end.\verseline
			The sled and traveller stopped, the courier's feet\verseline
			Delayed, all friends shut out, the housemates sit\verseline
			Around the radiant fireplace, enclosed\verseline
			In a tumultuous privacy of storm.
	\end{stanza}
	\begin{stanza}
			%indent\verseline
			Come see the north wind's masonry.\verseline
			Out of an unseen quarry evermore\verseline
			Furnished with the tile, the fierce artificer\verseline
			Curves his white bastions with projected roof\verseline
			Round every windward stake, or tree, or door.\verseline
			Speeding, the myriad-handed, his wild work\verseline
			So fanciful, so savage, nothing cares he\verseline
			For number or proportion. Mockingly,\verseline
			On coop or kennel he hangs Parian wreaths;\verseline
			A swan-like form invests the hidden thorn;\verseline
			Fills up the farmer's lane from wall to wall,\verseline
			Maugre the farmer's sighs; and at the gate\verseline
			A tapering turret overtops the work.\verseline
			And when his hours are numbered, and the world\verseline
			Is all his own, retiring, as he were not,\verseline
			Leaves, when the sun appears, astonished Art\verseline
			To mimic in slow structures, stone by stone\verseline
			Built in an age, the wind's night-work,\verseline
			The frolic architecture of the snow.
	\end{stanza}
\end{poem}
\articlebio{Ralph Waldo Emerson was a transcendentalist American poet, who lived from 1803 to 1882. A staunch opponent of the south in the civil war, Emerson championed a strong relationship with the land, in all its flora and fauna.}
\articlerights{Public Domain}