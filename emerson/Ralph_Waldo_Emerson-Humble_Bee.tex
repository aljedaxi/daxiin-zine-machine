THE HUMBLE-BEE
Ralph Waldo Emerson

Burly, dozing humble-bee,
Where thou art is clime for me.
Let them sail for Porto Rique,
Far-off heats through seas to seek;
I will follow thee alone,
Thou animated torrid-zone!
Zigzag steerer, desert cheerer,
Let me chase thy waving lines;
Keep me nearer, me thy hearer,
Singing over shrubs and vines.

Insect lover of the sun,
Joy of thy dominion!
Sailor of the atmosphere;
Swimmer through the waves of air;
Voyager of light and noon;
Epicurean of June;
Wait, I prithee, till I come
Within earshot of thy hum,--
All without is martyrdom.

When the south wind, in May days,
With a net of shining haze
Silvers the horizon wall,
And with softness touching all,
Tints the human countenance
With a color of romance,
And infusing subtle heats,
Turns the sod to violets,
Thou, in sunny solitudes,
Rover of the underwoods,
The green silence dost displace
With thy mellow, breezy bass.

Hot midsummer's petted crone,
Sweet to me thy drowsy tone
Tells of countless sunny hours,
Long days, and solid banks of flowers;
Of gulfs of sweetness without bound
In Indian wildernesses found;
Of Syrian peace, immortal leisure,
Firmest cheer, and bird-like pleasure.

Aught unsavory or unclean
Hath my insect never seen;
But violets and bilberry bells,
Maple-sap and daffodels,
Grass with green flag half-mast high,
Succory to match the sky,
Columbine with horn of honey,
Scented fern, and agrimony,
Clover, catchfly, adder's-tongue
And brier-roses, dwelt among;
All beside was unknown waste,
All was picture as he passed.

Wiser far than human seer,
Yellow-breeched philosopher!
Seeing only what is fair,
Sipping only what is sweet,
Thou dost mock at fate and care,
Leave the chaff, and take the wheat.
When the fierce northwestern blast
Cools sea and land so far and fast,
Thou already slumberest deep;
Woe and want thou canst outsleep;
Want and woe, which torture us,
Thy sleep makes ridiculous.
